\section{Related Works}
Modeling research performance has been explored in many different approaches, especially in modeling indicators of future research performance and predicting these indicators over time. Studies on research performance prediction offer research organizations more general ways to evaluate the potential research performance of a researcher in coming years.

Conventionally, the influence and popularity of publications are measured by the number of citations, typically by overall total citations for influence and yearly citations for recent popularity. Extensive works have been done on predicting citations of publications.

Total number of citations of an individual publication prediction based on historical citations has been studied in 2003 ACM SIGKDD Competition \cite{gehrke2003overview}. Predicting total number of citations is categorized in social network mining. Perlich and Provost proposed a statistical model technique with aggregation operators that capture more information about the value distributions, by storing meta-data about value distributions and referencing this meta-data when aggregating\cite{perlich2003aggregation}. Predicting the total citations of an individual publication will provide a general sense of the lifetime influence of the given publication but barely reflect its recent popularity.

A citation count prediction learning framework of publications is proposed by Yan et al.\cite{yan2011citation}. In their work, several publication related features alongside author features are chosen to predict the number of citations of publications in the next fixed number of years. They achieved a regression coefficient of determination($R^2$) at 0.75 for predicting citations in the next 5 years, with a classification and regression tree(CART) model proposed by Breimann in 1984\cite{breiman1984classification}. CART is tested out to performance better in this regression problem.

Metrics for evaluating the performance of researchers are introduced over the years. Conventionally, the total number of citations of all publications from an author is used to indicate his or her research performance. However, a researcher lacks productivity may benefit from a well cited publication, which means even though total citations reflects the popularity of a researcher, it does not necessarily reflect the productivity of researchers.
On the contrary, total number of publications published by a researcher reflects the productivity but does not reflect the popularity of a researcher.

Hirsch introduced the h-index, which reflects both the popularity and the productivity of researchers\cite{hirsch2005index}. He also reported that h-index is a more predictive indicator than number of citations and number of publications. H-index, when utilized to predict number of citations and papers in the future, will perform better than number of citations and papers themselves respectively\cite{hirsch2007does}. However, despite its widespread positive reception, some objections are proposed, majorly arguing the multidimensional researchers might suffer from a lower h-index and the performance of researchers should not be evaluated using a single metric\cite{van2006comparison,glanzel2006h,glanzel2006opportunities}.

Google also proposed the i10-index for their Google Scholar service, which counts the number of publications that have a citation count higher than or equal to $10$. i10-index is straight forward but does not used anywhere else except Google Scholar.

With both these index and total citations accepted as a general measurement for research performance, researchers are eager to publish more publications with a higher number of citations.
