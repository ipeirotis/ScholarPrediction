\section{Problem Definition}
In this section, we will formally define the prediction problem that we solve.
As has been discussed before, predicting total number of citations in the future does not necessarily reflect the recent production and popularity of researchers. Therefore, predicting the number of citations during a given time frame of researchers will give a better sense of the recent productivity and popularity of researchers.

In this paper, we will generalize a new publication-level approach to predict the performance of a researcher in the near future.

Following Yan et al.'s definition\cite{yan2011citation}, the citation of a publication $p$ in a collection of publications $P$ is defined as follows.

\begin{definition}
Given a publication collection $P$, and a publication $p\in P$, the citations of a publication $p$ at time $t$
\[C_T(p,t):=|citing(p,t)|,\]
where
\[citing(p):=\{\forall p'\in P, p' \textit{ cites } p \]
\[\textit{ AND } p' \textit{ is published before } t \}.\]
\end{definition}

We define the number of citations of a researcher as the aggregation of citations of all his or her publications.

\begin{definition}
The number of citations of a researcher $r$ at time(typically year) $t$ in a publication collection $P$
\[N_c(r,t):=\sum_{\forall p \in P(r)} C_T(p,t),\]
where
\[P(r):=\{\forall p \in P, r \textit{ coauthored } p\}.\]
\label{def-author-citations}
\end{definition}

In our perspective, the predicting of citations that is accumulated only in the next few years
will have a stronger representation of a researcher's activeness and potential research performance.
Therefore, our learning task is now defined as follows.

\begin{definition}
Given a set of researcher features, $\vec{X}=x_1,x_2,\cdots,x_n$, our goal is to learn a predictive function $f$ to predict the citation
of a research between time $t$ and some time period $\delta t$ later. Define
\[D_c(r,t,\Delta t):=N_c(r,t+\Delta t)-N_c(r,t)\]
%difference in citations
Formally, we are giving
\[f(r|\vec{X},t,\Delta t)\rightarrow D_c(r,t,\Delta t).\]
\end{definition}
