\section{Introduction}
Scientific research is becoming more and more influencing and has been creating a large collection of publications, especially in the past 50 years.
Google Scholar service went online on November 20 2004, and is estimated to hold the metadata for roughly 8.7 million publications as of May 2016\cite{}.
The growing size of publications over years is shown in figure \ref{publicationcount-year}.
\begin{figure}
\begin{center}
\resizebox {\columnwidth} {!} {

\begin{tikzpicture}
\begin{axis}[
	xlabel = Year,
	ylabel = Number of publication,
]

\addplot[color=blue] table[x={year}, y={count}]{data/publication-year};
\end{axis}
\end{tikzpicture}
}
\caption{Number of publications over year on Google Scholar.}
\label{publicationcount-year}
\end{center}
\end{figure}

Predicting the impact of a publication meets the needs of researchers. The impact of a publication is conventionally represented by the number of citations over the years.
However, research institutions and organizations are more interested on predicting and evaluating the overall performance of a researcher in the next several years.

The influence of a researcher may result from multiple publication from wide range of fields or might from a single milestone publication that revolutionized the whole field as well. However, the influence is time variant. The popularity of a researcher will change over time. This might result from the popularity of a research field or a new theory come into the filed.
The most direct and widely adopted evaluation of a researcher's scholar influence is his or her total citation counts. The number of citations and h-index can reflect the scholar and performance of a researcher in a lifetime scale\cite{bornmann2007we}. However, these indicators does not necessarily reflect a researchers recent and current scholar performance. Therefore, the number of citations in a short period of time is a better indicator of the recent scholar performance and influence of a researcher.

Since the citations of publications shows a power law distribution, traditional regressions commonly gives poor results from publication citation data\cite{cheng2014can,radicchi2008universality}. 

In our work, future citations counts are used as an indicator for research performance and popularity of a researcher. Our prediction task is to formulate a regression model for the number of citations a researcher get during a given time frame, based on the historical citation data and other author related features. We will present a unique two stage approach to formulate a prediction model on future citation counts for researchers. As we shall see in our work, the predicted future citations of publications will contribute a positive factor in predicting future citation counts for a researcher.

To give a better prediction, a regression model for individual publications is first formulated, which is majorly based on Yan et al's work\cite{yan2011citation}. This model is then utilized to generate a predicted future citations of a researcher based on all publications he or she published before the given time.