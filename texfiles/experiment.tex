\section{Experiments}
\subsection{Data Source}
We perform the researcher future citation prediction on the Google Scholar database.
It covers *** publications by *** researchers for more than 50 years.
The total number of citations is roughly *** in this data set.

We used all papers with cross validation to formulate the publication level future citation prediction model.
With this model we created a new feature that aggregates all the number of citation predicted from all papers a researcher published before year $t$.

For each researcher who has been active for at least $\Delta t$ years, we generate a new record in a new data set for each year after the first $\Delta t$ years, i.e. if a researcher has been active for 40 years and $\Delta t=5$. 35 records are generated for every year after five years in the field. 

Since our work is time independent, therefore, we randomly split the data with 50\% for training, 20\% validation and another 30\% for test. 

We compare our predicted future citations with actual future citations in the test set.

\subsection{Evaluation}
The coefficient of determination ($R^2$) is used in evaluating how well prediction model fits the test data.
The definition of $R^2$ is:
\[R^2=\frac{\sum_{r \in }\hat{D_c}(r,t,\Delta t)-\bar{D_c}}{}\]
where $\hat{D_c}$ is the predicted future citation for researcher $r$ in year $t$, 
and $\bar{D_c}$ represents the mean of the actual citations from year $t$ to $t+\Delta t$ for all researchers.

By definition, $R^2\in [0,1]$, with $R^2=0$ means the predicted citations does not fit the actual citations at all,
and $R^2=1$ means the model perfectly predicted the future citations. A model with a larger $R^2$ indicates a better prediction performance and is therefore desired. 
\subsection{Prediction Performance}
Following Yan et al.'s work, our prediction model for publication future citation is summarized in table \ref{tbl-pubprediction}.

\begin{table}[ht]
\begin{center}
\begin{tabular}{c|c|c|c}
\hline
$R^2$& LR  & RF & Ridge\\
\hline
\hline
$\Delta t=3$ &0.911 &0.945 &0.910\\
\hline
$\Delta t=5$ & 0.788 & 0.982 &0.789\\
\hline
\end{tabular}
\end{center}
\caption{The performance of publication future citation prediction by various models on test set.}
\label{tbl-pubprediction}

\end{table}

The prediction for researcher future citations is summarized in table \ref{tbl-r2}.
The best predictive performance for next five year researcher citations is plotted in figure \ref{researcher-prediction}.


\begin{figure}
\includegraphics[width=3 in]{fig/researcher-prediction.png}
\caption{Ridge Regression on researcher next five years prediction.}
\label{researcher-prediction}
\end{figure}

\begin{table*}[t]
\begin{center}
\begin{tabular}{c|c|c|c|c|c|c|c|c|c|c}
\hline
& \multicolumn{5}{|c|}{\textbf{$\Delta t =3$}}& \multicolumn{5}{|c}{$\Delta t =5$}\\

\hline
Methods & LR & RF & PR & BDT & Ridge & LR & RF & PR & BDT & Ridge\\
\hline\hline
+org.Rank &-0.001 &-0.001 &0.000 &0.000 &-0.001 &0.004 &0.017 & 0.004 & 0.017 &0.004\\
\hline
+$N_c(r,t)$ & 0.791& 0.740& -4.161&0.727 &0.756 & 0.657 &	0.590 & -1.104 &0.570 &0.657\\
\hline
+yearly citation &0.811 &0.790 &0.403 &0.785 &0.811 & 0.651 &0.613 & -1.933 & 0.635 & 0.718\\
\hline
+index & 0.459&0.404 &0.070 &0.447 &0.459 &0.392 &0.318 &-1.915 & 0.372 &0.392\\
\hline
+years &0.075 &0.075 & 0.060&0.076 &0.075 &0.058 & 0.066 &0.040 & 0.066 & 0.058\\
\hline
+publication  &0.518 &0.832 &0.108 &0.822 & 0.518 & 0.614 &0.883 & -0.074 & 0.896 & 0.614\\
\hline\hline
-org.Rank &0.804 &0.869 &0.123 & 0.847&0.822 & 0.809 & 0.956 & -1.052 &0.945 & 0.809\\
\hline
-$N_c(r,t)$ &0.823 &0.882 &-0.315 &0.846 & 0.823 &0.814 &	0.952 & -0.984&0.938 & 0.814\\
\hline
-yearly citation &0.804 &0.866 & 0.374& 0.831&0.777 &0.810 & 0.951&-0.059 &0.936 &0.800\\
\hline
-index &0.796 &0.868 &-0.146 &0.853 & 0.814&0.800 & 0.929&0.281 &0.922 &0.801\\
\hline
-years &0.801 &0.865 &0.187 &0.832 & 0.819 &0.803 &0.942 & -1.042 & 0.942 &0.804\\
\hline
-publication & 0.791& 0.826& 0.205& 0.787&0.812 & 0.725 & 0.798 & -2.021 & 0.789 & 0.726\\
\hline\hline
Combined &0.804 &0.867 &0.153 &0.851 &0.822 &0.809&\textbf{0.957} &-1.130 &0.946 &0.809\\
\hline
\end{tabular}
\end{center}
\caption{The performance of various models on test set. "+" stands for using only this feature. "-" stands for the drop of this feature group from all features.}
\label{tbl-r2}

\end{table*}



\subsubsection{Variance in Prediction}
As we can see, most of the variance in our predictive model are having a higher predicted future citations than actual. This may result from unpredictable situations, for example, the retirement of a researcher. 

Sidiropoulos et al. reported that h-index has shortcomings in what majorly of its inability to distinguish active and inactive(retired) researchers\cite{sidiropoulos2007generalized}. The higher h-index contributes as a positive factor in our prediction and results in a higher prediction. In our work this also reflects the potential lack of knowledge of the recent status of researchers.
{\sc still need explanation.}

\subsubsection{Citation Distribution}
As is shown in figure \ref{researcher-prediction}, the distribution of future researchers citations shows a long tail. 
Most researchers gets a total future citations less than 2,500 cites.

\subsubsection{Factor Contribution}
{\it Most Useful Feature}

{\it publication level prediction}

































