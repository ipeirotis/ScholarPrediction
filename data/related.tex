\section{Related Works}
Modeling Scientific performance has been explored in many different approaches, especially in modeling future indicators of scholar performance.
The research on scientific performance prediction for researchers offers organizations to evaluate the potential scholar performance of a researcher in the next few years.

Traditionally, the citation counts of a single publications is used as a measurement for scientific performance.
Hirsch introduced the h-index, which reflects both the popularity and the productivity of a researcher\cite{hirsch2005index}.
Google also proposed the i10-index for their Google Scholar service.
With both these index and total citations accepted as a general measurement for scholar performance of a researcher, researchers are eager to publish with a higher influence or citations.

In addition to measuring scientific performance many prediction model has been introduced.
The prediction of total number of citations of an individual publication based on historical citations has been studied in 2003 ACM SIGKDD Competition \cite{gehrke2003overview}.

{\it TO BE CONTINUED...}