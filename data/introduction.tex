\section{Introduction}
Scientific research is becoming more and more influencing and has been creating a large collection of publications, especially in the past 50 years.
Google Scholar service went online on November 20 2004, and is estimated to hold the metadata for roughly 160 million publications as of May 2014\cite{}.
The growing size of publications over years is shown in figure \ref{}.
Predicting the impact of a publication meets the needs of researchers. The impact of a publication is conventionally represented by the number of citations over the years.
However, research institutions and organizations are more interested on predicting and evaluating the overall performance of a researcher in the next several years.

The influence of a researcher may result from multiple publication from wide range of fields or might from a single milestone publication that revolutionized the whole field as well. However, the influence is time variant. The popularity of a researcher will change over time. This might result from the popularity of a research field or a new theory come into the filed.
The most direct and general evaluation of a researcher's scholar influence is total citation counts. The overall citations (and h-index), can reflect the scholar and performance of a researcher in a lifetime scale. However, these indicators does not necessarily reflect a researchers recent and current scholar performance. Therefore, the change and difference of the total citation counts of a researcher, that is the number of citations get in a few years, is a better indicator of the influence of a researcher.

Since the citations of publications shows a power law distribution. Traditional regressions commonly gives poor results from publication citation data\cite{cheng2014can,radicchi2008universality}. But, as we shall see in our work, the predicted future citations of publications will contribute a positive factor in predicting future citation counts for a researcher.

In our work, we formalize the prediction problem of future influence of a researcher. The future citations counts is used as an indicator for the influence and popularity of a researcher. Our prediction task is to formulate a regression model for the number of citations a researcher get during a certain time frame, based on the historical citation data other features. We will present a unique two stage approach to formulate a prediction model on future citation counts for researchers.

To give a better prediction, we first formulated a regression model for individual publications majorly based on Rui et al's work\cite{yan2011citation}. Then, we utilized this model to generate a predicted future citations of a researcher based on all publications he or she published before that time. As we can see in the next few sections, the number of this predicted citation counts is generally smaller than the actual citations. This should be result from the papers that is published in the years after the prediction time which are getting more citations than expected.